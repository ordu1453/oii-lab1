\chapter{Практическая часть}

\section{Модель базы знаний}
База знаний реализована с использованием фреймовой модели: каждый сорт пива представлен как фрейм с набором атрибутов (слотов).  

\subsection{Пример фрейма}

\begin{verbatim}
Фрейм: IPA
  Горечь: высокая
  Сладость: низкая
  Аромат: цитрусовый, хвойный, хмелевой
  Цвет: золотистый
  Крепость: сильная
  Страна: Великобритания
\end{verbatim}

\subsection{ER-диаграмма}
Ниже на рисунке \ref{fig:beer-er} представлена диаграмма сущностей и связей в нотации Чена:~
\newpage

\begin{figure}[h!]
\begin{center}
\begin{tikzpicture}[
  entity/.style={rectangle, draw=black, thick, minimum width=4cm, minimum height=1.2cm, align=center, fill=white},
  relationship/.style={diamond, draw=black, thick, minimum width=2.4cm, minimum height=1.2cm, align=center, fill=white},
  >=Stealth,
  every edge/.style={draw,->,>=Stealth},
  font=\large
]


% Центральная сущность
\node[entity] (beer) at (-2,0) {Пиво};

% Остальные сущности — увеличенные расстояния
\node[entity] (taste)      at (-8,-5) {Вкус};
\node[entity] (country)    at (5,-12)  {Страна};
\node[entity] (preference) at (-8,-12) {Предпочтение};
\node[entity] (color) at (-3,-5) {Цвет};
\node[entity] (hardness) at (2,-5) {Крепость};


\node[relationship] (has)      at (-8,-2.5) {имеет};
\node[relationship] (assoc)    at (5,-8)  {ассоциируется};
\node[relationship] (matches)  at (-8,-8.5)  {соответствует};


% Пиво → Вкус
\draw (beer.south) -- ++(0,-0.4) -| (has.north);
\draw (has.south) -- ++(0,-0.4) -| (taste.north);
\draw (has.south) -- ++(0,-0.4) -| (color.north);
\draw (has.south) -- ++(0,-0.4) -| (hardness.north);


% Пиво → Страна
\draw (beer.south) -- ++(0,-0.4) -| (assoc.north);
\draw (assoc.south) -- ++(0,-0.4) -| (country.north);

% Предпочтение → Вкус
\draw (preference.north) -- ++(0,0.4) -- (matches.south);
\draw (matches.north) -- ++(0,0.4) -| (taste.south);
\draw (matches.north) -- ++(0,0.4) -| (color.south);
\draw (matches.north) -- ++(0,0.4) -| (hardness.south);



\end{tikzpicture}
\end{center}
\caption{ER-диаграмма предметной области «Сорта пива»}
\label{fig:beer-er}
\end{figure}


\section{Типовой сценарий взаимодействия}
Пользователь выбирает один из пунктов меню:
\begin{enumerate}
    \item Показать характеристики пива
    \item Рекомендовать пиво по вкусу
    \item Найти пиво по стране происхождения
    \item Найти пиво по крепости и цвету
\end{enumerate}

Пример использования:
\begin{verbatim}
Выбор пункта меню: Рекомендовать пиво по вкусу
Выбор горечи: средняя
Выбор аромата: фруктовый
Результат: Эль, Американский Эль
\end{verbatim}

\section{Пример работы программы}
\begin{verbatim}
=== Экспертная система: Вкусы пива ===
1. Показать характеристики пива
2. Рекомендовать пиво по вкусу
3. Найти по стране происхождения
4. Найти по крепости и цвету
0. Выход
Выберите действие: 2

Выберите желаемую горечь:
1. низкая
2. средняя
3. высокая
Номер горечи: 2

Выберите желаемый аромат:
1. фруктовый
2. цветочный
3. шоколадный
...
Номер аромата: 1

Рекомендуется: Эль, Американский Эль
\end{verbatim}

\section{Заключение}
В ходе лабораторной работы была разработана экспертная система по сортам пива с фреймовой базой знаний.  
Система позволяет:
\begin{itemize}
    \item хранить информацию о более чем 20 сортах пива с различными характеристиками;
    \item делать рекомендации по вкусовым предпочтениям, стране происхождения и крепости/цвету;
    \item демонстрировать атрибуты конкретного сорта пива пользователю.
\end{itemize}

Дальнейшее развитие системы может включать расширение базы знаний, подключение пользовательских отзывов, а также интеграцию с графическим интерфейсом.
