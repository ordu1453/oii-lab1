\chapter{Практическая часть}

\section{Особенности программной реализации}

Программная реализация экспертной системы выполнена на языке Python и представляет собой интерактивное консольное приложение. Основой системы служит база знаний \texttt{base}, содержащая характеристики различных сортов пива: тип горечи, аромат, цвет, крепость и страну происхождения (замечу, что при опросе эксперта данный пункт фигурировал как 'страна с которой ассоциируется этот сорт пива', поэтому корректность именования этого атребута 'страной происхождения' подлежит сомнению). 

Логика взаимодействия с пользователем реализована через меню, предоставляющее четыре основных режима работы: вывод характеристик выбранного сорта, рекомендация пива по вкусовым параметрам, поиск по стране происхождения и поиск по сочетанию крепости и цвета. Каждый пункт меню связан с отдельной функцией, которая осуществляет обработку пользовательского ввода, фильтрацию данных из базы знаний и вывод результата. 


\section{Модель базы знаний}
База знаний реализована с использованием фреймовой модели: каждый сорт пива представлен как фрейм с набором атрибутов.  

\includelistingpretty
    {frame.py} % Имя файла с расширением (файл должен быть расположен в директории inc/lst/)
    {} % Язык программирования (необязательный аргумент)
    {Пример объекта в базе данных} % Подпись листинга


\section{Типовой сценарий взаимодействия}
Пользователь выбирает один из пунктов меню:
\begin{enumerate}
    \item Показать характеристики пива
    \item Рекомендовать пиво по вкусу
    \item Найти пиво по стране происхождения
    \item Найти пиво по крепости и цвету
\end{enumerate}

На рисунке \ref{fig:beer-usecase} предствалена Use case диаграмма программы. 
В листинге \ref{lst:out.txt} представлен пример работы с программой.

\newpage
\begin{figure}[h!]
\centering
\begin{tikzpicture}[
    actor/.style={draw, thick, fill=white, minimum width=1.5cm, minimum height=0.8cm},
    usecase/.style={ellipse, draw, thick, minimum width=4.2cm, minimum height=1.2cm, align=center, fill=white},
    line/.style={draw, thick},
    font=\large
]

% Актор
\node[actor] (user) at (-7,0) {Пользователь};

% Use-cases
\node[usecase] (show_all)      at (2.5, 3) {Просмотр всех сортов пива};
\node[usecase] (by_taste)      at (2.5, 1) {Поиск по вкусу};
\node[usecase] (by_strength)   at (2.5,-1) {Поиск по крепости};
\node[usecase] (by_country)    at (2.5,-3) {Поиск по стране};
\node[usecase] (details)       at (2.5,-5) {Просмотр характеристик сорта};

% Линии связи
\draw[line] (user) -- (show_all);
\draw[line] (user) -- (by_taste);
\draw[line] (user) -- (by_strength);
\draw[line] (user) -- (by_country);
\draw[line] (user) -- (details);

\end{tikzpicture}
\caption{Use Case диаграмма}
\label{fig:beer-usecase}
\end{figure}



\includelistingpretty
    {out.txt} % Имя файла с расширением (файл должен быть расположен в директории inc/lst/)
    {} % Язык программирования (необязательный аргумент)
    {Пример работы с консольной программой} % Подпись листинга


\section{ER-диаграмма}
Ниже на рисунке \ref{fig:beer-er} представлена диаграмма сущностей и связей в нотации Чена:
% \newpage

\begin{figure}[h!]
\begin{center}
\begin{tikzpicture}[
  entity/.style={rectangle, draw=black, thick, minimum width=4cm, minimum height=1.2cm, align=center, fill=white},
  relationship/.style={diamond, draw=black, thick, minimum width=2.4cm, minimum height=1.2cm, align=center, fill=white},
  usecase/.style={ellipse, draw, thick, minimum width=4.2cm, minimum height=1.2cm, align=center, fill=white},
  >=Stealth,
  every edge/.style={draw,->,>=Stealth},
  font=\large
]


% Центральная сущность
\node[entity] (beer) at (-2,0) {Пиво};

% Остальные сущности — увеличенные расстояния
\node[usecase] (taste)      at (-8,-5) {Вкус};
\node[usecase] (country)    at (5,-12)  {Страна};
\node[entity] (preference) at (-8,-12) {Предпочтение};
\node[usecase] (color) at (-3,-5) {Цвет};
\node[usecase] (hardness) at (2,-5) {Крепость};


\node[relationship] (has)      at (-8,-2.5) {имеет};
\node[relationship] (assoc)    at (5,-8)  {ассоциируется};
\node[relationship] (matches)  at (-8,-8.5)  {соответствует};


% Пиво → Вкус
\draw (beer.south) -- ++(0,-0.4) -| (has.north);
\draw (has.south) -- ++(0,-0.4) -| (taste.north);
\draw (has.south) -- ++(0,-0.4) -| (color.north);
\draw (has.south) -- ++(0,-0.4) -| (hardness.north);


% Пиво → Страна
\draw (beer.south) -- ++(0,-0.4) -| (assoc.north);
\draw (assoc.south) -- ++(0,-0.4) -| (country.north);

% Предпочтение → Вкус
\draw (preference.north) -- ++(0,0.4) -- (matches.south);
\draw (matches.north) -- ++(0,0.4) -| (taste.south);
\draw (matches.north) -- ++(0,0.4) -| (color.south);
\draw (matches.north) -- ++(0,0.4) -| (hardness.south);


\end{tikzpicture}
\end{center}
\caption{ER-диаграмма предметной области}
\label{fig:beer-er}
\end{figure}



% \begin{figure}[h!]
% \centering
% \begin{tikzpicture}[
%   sibling distance=12em,
%   % стиль для круглых узлов
%   roundnode/.style={draw, circle, minimum size=1.2cm, align=center},
%   % стиль для квадратных узлов
%   squarenode/.style={draw, rectangle, minimum size=1.2cm, align=center}
% ]

% \node[squarenode] {Пиво}  % корневой узел круглый
%     child { node[squarenode] {Вкус}  % квадратный узел
%       child { node[squarenode] {Горечь} }
%       child { node[squarenode] {Сладость}}
%       child { node[squarenode] {Аромат}}
%     }
%     child { node[squarenode] {Страна}  % квадратный узел
%       child { node[roundnode] {Германия} }
%       child { node[roundnode] {...} }
%     }
%     child { node[squarenode] {Цвет}  % квадратный узел
%       child { node[roundnode] {D} }
%       child { node[roundnode] {E} }
%     }
%     child { node[squarenode] {Крепость}  % квадратный узел
%       child { node[roundnode] {D} }
%       child { node[roundnode] {E} }
%     };

% \end{tikzpicture}
% \caption{Use-Case диаграмма системы}
% \label{fig:beer-usecase}
% \end{figure}




\chapter*{ЗАКЛЮЧЕНИЕ}
В ходе лабораторной работы была разработана экспертная система по сортам пива с фреймовой базой знаний.  
Система позволяет:
\begin{itemize}
    \item хранить информацию сортах пива с различными характеристиками;
    \item делать рекомендации по вкусовым предпочтениям, стране происхождения и крепости/цвету;
    \item демонстрировать атрибуты конкретного сорта пива пользователю.
\end{itemize}

