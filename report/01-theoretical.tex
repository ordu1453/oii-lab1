\chapter{Теоретическая часть}

\section{Введение}
Цель лабораторной работы: создание экспертной системы для работы с базой знаний о вкусах и характеристиках различных сортов пива.  
Задачи работы:
\begin{itemize}
    \item Провести опрос эксперта о вкусах пива и собрать базу знаний.
    \item Реализовать фреймовую базу знаний с различными характеристиками сортов пива.
    \item Создать консольное приложение для взаимодействия с базой знаний.
    \item Реализовать не менее трёх запросов к системе с параметрами.
\end{itemize}

\section{Описание предметной области}
Предметная область включает информацию о сортах пива и их характеристиках:
\begin{itemize}
    \item \textbf{Типы пива:} Эль, Лагер, Стаут, IPA, Пшеничное, Портер, Трипель, Бельгийский Витбир, Светлый Эль, Дабл IPA, Кёльш, Бок, Пильзенер, Американский Эль, Светлый Ламбик, Овсяный Стаут, Барливайн, Ред Эль.
    \item \textbf{Характеристики:} горечь, сладость, аромат, цвет, крепость (слабая/средняя/сильная), страна.
\end{itemize}

На основе этих знаний система может рекомендовать сорта пива по выбранным пользователем параметрам.
