\chapter{Теоретическая часть}

\section{Введение}
Цель лабораторной работы: создание программы для работы с базой знаний о вкусах и характеристиках различных сортов пива.  
Задачи работы:
\begin{itemize}
    \item Провести опрос эксперта о вкусах пива и собрать базу знаний.
    \item Реализовать базу знаний с различными характеристиками сортов пива.
    \item Создать приложение для взаимодействия с базой знаний.
    \item Реализовать не менее трёх запросов к системе с параметрами.
\end{itemize}

\section{Описание предметной области}
Предметная область включает информацию о сортах пива и их характеристиках. На рисунке \ref{img:tree} представлена схема рассматриваемой предметной области.
\includeimage
    {tree} % Имя файла без расширения (файл должен быть расположен в директории inc/img/)
    {f} % Обтекание (без обтекания)
    {h} % Положение рисунка (см. figure из пакета float)
    {0.8\textwidth} % Ширина рисунка
    {Схема предметной области} % Подпись рисунка

На основе этих знаний система может рекомендовать сорта пива по выбранным пользователем параметрам.
